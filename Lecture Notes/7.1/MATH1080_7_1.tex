%%%%%%%%%%%%%%%%%%%%%%%%%%%%%%%%%%%%%%%%%
% Class Notes Template
% LaTeX Template
% By: Ryan Grove
%%%%%%%%%%%%%%%%%%%%%%%%%%%%%%%%%%%%%%%%%

%----------------------------------------------------------------------------------------
%	PACKAGES AND OTHER DOCUMENT CONFIGURATIONS
%----------------------------------------------------------------------------------------

\documentclass[paper=a4, fontsize=11pt]{scrartcl} % A4 paper and 11pt font size

\usepackage[T1]{fontenc} % Use 8-bit encoding that has 256 glyphs
\usepackage{fourier} % Use the Adobe Utopia font for the document - comment this line to return to the LaTeX default
\usepackage[english]{babel} % English language/hyphenation
\usepackage{amsmath,amsfonts,amsthm} % Math packages

\usepackage{lipsum} % Used for inserting dummy 'Lorem ipsum' text into the template

\usepackage{sectsty} % Allows customizing section commands
\allsectionsfont{\centering \normalfont\scshape} % Make all sections centered, the default font and small caps

\usepackage{fancyhdr} % Custom headers and footers
\pagestyle{fancyplain} % Makes all pages in the document conform to the custom headers and footers
\fancyhead{} % No page header - if you want one, create it in the same way as the footers below
\fancyfoot[L]{} % Empty left footer
\fancyfoot[C]{} % Empty center footer
%\fancyfoot[R]{\thepage} % Page numbering for right footer
\renewcommand{\headrulewidth}{0pt} % Remove header underlines
\renewcommand{\footrulewidth}{0pt} % Remove footer underlines
\setlength{\headheight}{13.6pt} % Customize the height of the header

\numberwithin{equation}{section} % Number equations within sections (i.e. 1.1, 1.2, 2.1, 2.2 instead of 1, 2, 3, 4)
\numberwithin{figure}{section} % Number figures within sections (i.e. 1.1, 1.2, 2.1, 2.2 instead of 1, 2, 3, 4)
\numberwithin{table}{section} % Number tables within sections (i.e. 1.1, 1.2, 2.1, 2.2 instead of 1, 2, 3, 4)

\setlength\parindent{0pt} % Removes all indentation from paragraphs - comment this line for an assignment with lots of text

\usepackage{lastpage}
\usepackage{fancyhdr}
\cfoot{\thepage\ of \pageref{LastPage}}

\def\v{\hbox{$\mathbf v$}}
\def\w{\hbox{$\mathbf w$}}
\def\u{\hbox{$\mathbf u$}}
\def\x{\hbox{$\textbf{x}$}}
\def\z{\hbox{$\mathbf z$}}
\def\a{\hbox{$\mathbf a$}}
\def\b{\hbox{$\mathbf b$}}
\def\L{\hbox{$\mathcal L$}}
\def\C{\hbox{$\mathbb C$}}
\def\B{\hbox{$\mathcal B$}}
\def\R{\hbox{$\mathbb R$}}
\def\X{\hbox{$\underline X$}}
\def\Q{\hbox{$\mathbb Q$}}
\def\R{\hbox{$\mathbb R$}}
\def\N{\hbox{$\mathbb N$}}
\def\C{\hbox{$\mathbb C$}}
\def\0{\hbox{$\mathbf 0$}}
\def\Y{\hbox{$\underline Y$}}
\def\a{\hbox{$\mathbf a$}}
\def\u{\hbox{$\mathbf u$}}
\def\w{\hbox{$\mathbf w$}}
\def\y{\hbox{$\mathbf y$}}
\def\X{\hbox{$\underline X$}}
\def\dd{\hbox{$\partial $}}
\def\B{\hbox{$\mathcal B$}}
\def\F{\hbox{$\mathcal F$}}
\def\L{\hbox{$\mathcal L$}}
\def\M{\hbox{$\mathcal M$}}
\def\D{\hbox{$\mathscr {D}$}}
\def\RR{\hbox{$\mathscr{R}$}}
\def\I{\hbox{$\mathcal I$}}

\usepackage{amssymb}
%\theoremstyle{plain}
\usepackage[margin = .75in]{geometry}
\newtheorem{claim}{Claim}
\newtheorem{theorem}{Theorem}[section]
\newtheorem{lemma}[theorem]{Lemma}
\newtheorem{proposition}[theorem]{Proposition}
\newtheorem{corollary}[theorem]{Corollary}
\newtheorem{problem}[theorem]{Problem}
%\theoremstyle{definition}
\newtheorem{definition}[theorem]{Definition}
%\theoremstyle{remark}
\newtheorem{remark}[theorem]{Remark}
\newtheorem{remarks}[theorem]{Remarks}
\newtheorem{example}[theorem]{Example}
\newcommand{\ds}{\displaystyle}
\newcommand{\ZZ}{\mathbb{Z}}
\newcommand{\QQ}{\mathbb{Q}}
\newcommand{\e}{\varepsilon}
\newcommand{\bbf}{\textbf}
\newcommand{\p}{\parallel}
\usepackage{color}
\newcommand{\field}[1]{\mathbb{#1}}
\usepackage{amsmath}
\usepackage{amsthm}
\usepackage{amssymb}
\usepackage{mathrsfs}
\usepackage{cancel}
\usepackage{upgreek}
\usepackage{graphicx}
\usepackage{multirow}
\usepackage{setspace}
\usepackage{url}
\usepackage{subfigure}
\usepackage{enumerate}
\usepackage{cases}
\usepackage{mathrsfs}
\usepackage{rotating}

%----------------------------------------------------------------------------------------
%	TITLE SECTION
%----------------------------------------------------------------------------------------

\newcommand{\horrule}[1]{\rule{\linewidth}{#1}} % Create horizontal rule command with 1 argument of height

\title{	
\normalfont \normalsize 
\textsc{Ryan Grove, Clemson University, MATH1080 - 9} \\ [25pt] % Your name, university, class
\horrule{0.5pt} \\[0.4cm] % Thin top horizontal rule
\huge Section 7.1: Integration by Parts \\ % The assignment title
\horrule{2pt} \\[0.5cm] % Thick bottom horizontal rule
}

\author{Date:} % The due date

\date{\normalsize January ???, 2016} % A custom date

\begin{document}

\maketitle % Print the title

\begin{flushleft}
\begin{tabular}{l l}
Name: \rule{3.2in}{.01cm}  & {}%Table number: \rule{1in}{.01cm}\\
\end{tabular}
\end{flushleft}

%----------------------------------------------------------------------------------------
%	Lecture
%----------------------------------------------------------------------------------------

\section*{\textbf{Lecture:}}

Every differentiation rule has a corresponding integration rule. For instance the Substitution Rule for integration corresponds to the Chain Rule for differentiation. The rule that corresponds to the Product Rule for differentiation is called the rule for \underline{\hspace{1.25in}} \underline{\hspace{0.4in}} \underline{\hspace{0.8in}}.\\
\indent

The Product Rule states that if $f$ and $g$ are differentiable functions, then\\

\[\ds\frac{d}{dx}[f(x)g(x)] = \underline{\hspace{2.5in}}\]
\indent

In the notation for indefinite integrals this equation becomes

\begin{align*}
&\ds\int [f(x)g'(x) + g(x)f'(x)]dx = f(x)g(x)\\
\implies \quad \quad &\ds\int f(x)g'(x)dx + \ds\int g(x)f'(x)dx = f(x)g(x)\\
\text{ } &\\
\implies \quad \quad &\boxed{\ds\int f(x)g'(x)dx = f(x)g(x) - \ds\int g(x)f'(x)dx} \quad \quad [1]
\end{align*}

Formula 1 is called the \textbf{formula for integration by parts}. It is perhaps easier to remember in the following notation. Let $u=f(x)$ and $v=g(x)$. Then the differentials are $du=f'(x)dx$ and $dv=g'(x)dx$, so, by the Substitution Rule, the formula for integration by parts becomes:

\indent\\
\indent\\
\[\underline{\hspace{3in}} \hspace{1in} [2]\]
%\[\boxed{\quad \ds\int u dv = uv - \ds\int v du \quad } \quad \quad [2]\]
\indent\\
\newpage
\underline{Example 1}: Find $\ds\int x\sin x dx$.\\
\indent

\textbf{Solution using Formula 2:}\\
\indent

\vspace{2in}


\textbf{Note:} Our aim in using integration by parts is to obtain a simpler integral than the one we started with. For instance, in Example 1, we started with $\ds\int x\sin x dx$ and expressed it in terms of the simpler integral $\ds\int \cos x dx$. If we had instead chosen $u=\sin x$ and $dv = x dx$, then

\[\text{ }\]
so integration by parts gives

\[\text{ }\]
\indent

Although this is true, $\ds\int x^2\cos x dx$ is a more difficult integral than the one we started with. In general, when deciding on a choice for $u$ and $dv$, we usually try to choose $u=f(x)$ to be a function that becomes simpler when differentiated (or at least not more compicated) as long as $dv = g'(x)dx$ can be readily integrated to give $v$.\\
\indent

One known strategy or rule of thumb for choosing $u$ is to follow the acronym \underline{\hspace{1in}}:
\begin{align*}
\text{I } &- \hspace{4in}\\
\text{ } &\\
\text{L } &- \hspace{4in}\\
\text{ } &\\
\text{A } &- \hspace{4in}\\
\text{ } &\\
\text{T } &- \hspace{4in}\\
\text{ } &\\
\text{E } &- \hspace{4in}
\end{align*}

*Typically, whichever function comes \underline{\hspace{1in}} in the list above should be \underline{\hspace{0.4in}} and whichever comes \underline{\hspace{1in}} in the list should be \underline{\hspace{0.5in}} since functions lower in the list have easier antiderivatives.\\
\indent
\newpage
\underline{Example 2}: Evaluate $\ds\int \ln x dx$.\\
\indent

\vspace{3in}


\underline{Example 3}: Find $\ds\int t^2 e^t dt$.\\
\indent

\vspace{3in}

\underline{Example 4}: Evaluate $\ds\int e^x\sin x dx$.\\
\indent

\vspace{4in}

\newpage

The formula for integration by parts for definite integrals is as follows:

\[\boxed{\quad \ds\int_a^b f(x) g'(x)dx = f(x) g(x)\bigg{|}_a^b - \ds\int_a^b g(x) f'(x)dx \quad }\]

or in terms of $u$ and $v$:

\[ \boxed{\quad \ds\int_a^b u dv = uv\bigg{|}_a^b - \ds\int_a^b v du \quad }\]
\indent\\
\indent\\
\indent

\underline{Example 5}: Calculate $\ds\int_0^1 \tan^{-1}x dx$.\\
\indent

\vspace{4in}


%----------------------------------------------------------------------------------------

\end{document}