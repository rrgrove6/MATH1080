%%%%%%%%%%%%%%%%%%%%%%%%%%%%%%%%%%%%%%%%%
% Class Notes Template
% LaTeX Template
% By: Ryan Grove
%%%%%%%%%%%%%%%%%%%%%%%%%%%%%%%%%%%%%%%%%

%----------------------------------------------------------------------------------------
%	PACKAGES AND OTHER DOCUMENT CONFIGURATIONS
%----------------------------------------------------------------------------------------

\documentclass[paper=a4, fontsize=11pt]{scrartcl} % A4 paper and 11pt font size

\usepackage[T1]{fontenc} % Use 8-bit encoding that has 256 glyphs
\usepackage{fourier} % Use the Adobe Utopia font for the document - comment this line to return to the LaTeX default
\usepackage[english]{babel} % English language/hyphenation
\usepackage{amsmath,amsfonts,amsthm} % Math packages

\usepackage{lipsum} % Used for inserting dummy 'Lorem ipsum' text into the template

\usepackage{sectsty} % Allows customizing section commands
\allsectionsfont{\centering \normalfont\scshape} % Make all sections centered, the default font and small caps

\usepackage{fancyhdr} % Custom headers and footers
\pagestyle{fancyplain} % Makes all pages in the document conform to the custom headers and footers
\fancyhead{} % No page header - if you want one, create it in the same way as the footers below
\fancyfoot[L]{} % Empty left footer
\fancyfoot[C]{} % Empty center footer
%\fancyfoot[R]{\thepage} % Page numbering for right footer
\renewcommand{\headrulewidth}{0pt} % Remove header underlines
\renewcommand{\footrulewidth}{0pt} % Remove footer underlines
\setlength{\headheight}{13.6pt} % Customize the height of the header

\numberwithin{equation}{section} % Number equations within sections (i.e. 1.1, 1.2, 2.1, 2.2 instead of 1, 2, 3, 4)
\numberwithin{figure}{section} % Number figures within sections (i.e. 1.1, 1.2, 2.1, 2.2 instead of 1, 2, 3, 4)
\numberwithin{table}{section} % Number tables within sections (i.e. 1.1, 1.2, 2.1, 2.2 instead of 1, 2, 3, 4)

\setlength\parindent{0pt} % Removes all indentation from paragraphs - comment this line for an assignment with lots of text

\usepackage{lastpage}
\usepackage{fancyhdr}
\cfoot{\thepage\ of \pageref{LastPage}}

\def\v{\hbox{$\mathbf v$}}
\def\w{\hbox{$\mathbf w$}}
\def\u{\hbox{$\mathbf u$}}
\def\x{\hbox{$\textbf{x}$}}
\def\z{\hbox{$\mathbf z$}}
\def\a{\hbox{$\mathbf a$}}
\def\b{\hbox{$\mathbf b$}}
\def\L{\hbox{$\mathcal L$}}
\def\C{\hbox{$\mathbb C$}}
\def\B{\hbox{$\mathcal B$}}
\def\R{\hbox{$\mathbb R$}}
\def\X{\hbox{$\underline X$}}
\def\Q{\hbox{$\mathbb Q$}}
\def\R{\hbox{$\mathbb R$}}
\def\N{\hbox{$\mathbb N$}}
\def\C{\hbox{$\mathbb C$}}
\def\0{\hbox{$\mathbf 0$}}
\def\Y{\hbox{$\underline Y$}}
\def\a{\hbox{$\mathbf a$}}
\def\u{\hbox{$\mathbf u$}}
\def\w{\hbox{$\mathbf w$}}
\def\y{\hbox{$\mathbf y$}}
\def\X{\hbox{$\underline X$}}
\def\dd{\hbox{$\partial $}}
\def\B{\hbox{$\mathcal B$}}
\def\F{\hbox{$\mathcal F$}}
\def\L{\hbox{$\mathcal L$}}
\def\M{\hbox{$\mathcal M$}}
\def\D{\hbox{$\mathscr {D}$}}
\def\RR{\hbox{$\mathscr{R}$}}
\def\I{\hbox{$\mathcal I$}}

\usepackage{amssymb}
%\theoremstyle{plain}
\usepackage[margin = .75in]{geometry}
\newtheorem{claim}{Claim}
\newtheorem{theorem}{Theorem}[section]
\newtheorem{lemma}[theorem]{Lemma}
\newtheorem{proposition}[theorem]{Proposition}
\newtheorem{corollary}[theorem]{Corollary}
\newtheorem{problem}[theorem]{Problem}
%\theoremstyle{definition}
\newtheorem{definition}[theorem]{Definition}
%\theoremstyle{remark}
\newtheorem{remark}[theorem]{Remark}
\newtheorem{remarks}[theorem]{Remarks}
\newtheorem{example}[theorem]{Example}
\newcommand{\ds}{\displaystyle}
\newcommand{\ZZ}{\mathbb{Z}}
\newcommand{\QQ}{\mathbb{Q}}
\newcommand{\e}{\varepsilon}
\newcommand{\bbf}{\textbf}
\newcommand{\p}{\parallel}
\usepackage{color}
\newcommand{\field}[1]{\mathbb{#1}}
\usepackage{amsmath}
\usepackage{amsthm}
\usepackage{amssymb}
\usepackage{mathrsfs}
\usepackage{cancel}
\usepackage{upgreek}
\usepackage{graphicx}
\usepackage{multirow}
\usepackage{setspace}
\usepackage{url}
\usepackage{subfigure}
\usepackage{enumerate}
\usepackage{cases}
\usepackage{mathrsfs}
\usepackage{rotating}

%----------------------------------------------------------------------------------------
%	TITLE SECTION
%----------------------------------------------------------------------------------------

\newcommand{\horrule}[1]{\rule{\linewidth}{#1}} % Create horizontal rule command with 1 argument of height

\title{	
\normalfont \normalsize 
\textsc{Ryan Grove, Clemson University, MATH1080 - 9} \\ [25pt] % Your name, university, class
\horrule{0.5pt} \\[0.4cm] % Thin top horizontal rule
\huge Section 7.4: Partial Fraction Decomposition \\ % The assignment title
\horrule{2pt} \\[0.5cm] % Thick bottom horizontal rule
}

\author{Date:} % The due date

\date{\normalsize February 5th, 2016} % A custom date

\begin{document}

\maketitle % Print the title

\begin{flushleft}
\begin{tabular}{l l}
Name: \rule{3.2in}{.01cm}  & {}%Table number: \rule{1in}{.01cm}\\
\end{tabular}
\end{flushleft}

%----------------------------------------------------------------------------------------
%	Lecture
%----------------------------------------------------------------------------------------

\section*{\textbf{Lecture:}}

In this section we show how to integrate any rational function (a ratio of polynomials) by expressing it as a sum of simpler fractions, called \underline{\hspace{1.25in}} \underline{\hspace{1.25in}}, that we already know how to integrate. To illustrate the method, observe that by taking the fractions $\ds\frac{2}{x-1}$ and $\ds\frac{1}{x+2}$ to a common denominator we obtain:

\[\ds\frac{2}{x-1} - \ds\frac{1}{x+2} = \hspace{3in}\]

\indent

So if we begin with the integral, $\ds\int \ds\frac{x+5}{x^2+x-2}dx$, we can reverse the above procedure to evaluate this integral as follows:
\begin{align*}
\ds\int \ds\frac{x+5}{x^2 +x-2}dx &= \ds\int\left(\ds\frac{2}{x-1} - \ds\frac{1}{x+2}\right)dx\\
&\text{ }\\
&= \hspace{2in}
\end{align*}

Consider a general rational function:
\[f(x) = \ds\frac{P(x)}{Q(x)}\]

where $P$ and $Q$ are polynomials.\\
\indent

Recall that if 
\[P(x) = a_n x^n + a_{n-1}x^{n-1} + \cdots + a_1 x + a_0\]

where $a_n\neq 0$, then the degree of $P$ is $n$ and we write deg$(P)=n$.\\
\indent

\underline{Two Scenarios}:\\
\indent

(1) deg$(P)<$ deg$(Q)$: \quad $\implies f(x)$ is \underline{\hspace{1.25in}}. \\
\begin{quote}
\flushleft
 Here it is possible to express $f$ as a sum of simpler fractions (i.e. \textit{partial fractions}).\\
\indent
\end{quote}

(2) deg$(P)\geq$ deg$(Q)$: \quad $\implies f(x)$ is \underline{\hspace{1.25in}}.\\
\begin{quote}
\flushleft
Here we must take the preliminary step of dividing $Q$ into $P$ by \underline{\hspace{1in}} \underline{\hspace{1.25in}} until a remainder $R(x)$ is obtained such that deg$(R)$ $<$ deg$(Q)$ (Scenario 1). The division statement is
\[f(x) = \ds\frac{P(x)}{Q(x)} = S(x) + \ds\frac{R(x)}{Q(x)}\]

where $S$ and $R$ are also polynomials. \\
\indent

*Sometimes this preliminary step is all that's needed. Other times, we then have to do partial fraction decomposition on the \textit{proper} rational function $\ds\frac{R(x)}{Q(x)}$ (Scenario 1).\\
\indent
\end{quote}

\underline{Example 1}: Find $\ds\int \ds\frac{x^3 + x}{x-1}dx$.\\
\indent

\vspace{3in}

\section*{Partial Fraction Decomposition}

For Scenario (1) above where $f(x)$ is \textit{proper}, i.e. deg$(P)<$ deg$(Q)$, we perform \textbf{Partial Fraction Decomposition (PFD)}:\\
\indent

\textbf{Step 1:} Factor the denominator $Q(x)$ as far as possible. Any polynomial $Q$ can be factored as a\\
\hspace{43pt} product of linear factors ($ax+b$) and irreducible quadratic factors ($ax^2 + bx + c$, where\\
\hspace{43pt} $b^2-4ac<0$). For example,\\

\[Q(x) = x^4 - 16 = \hspace{3.5in}\]
\indent\\

\textbf{Step 2:} Express the proper rational function $\ds\frac{P(x)}{Q(x)}$ [in case (1)] or $\ds\frac{R(x)}{Q(x)}$ [in Scenario (2)] as a \\
\hspace{47pt}sum of \textbf{partial fractions} of the form

\[\ds\frac{A}{(ax+b)^{m}} \quad \text{ or } \quad \ds\frac{Ax+B}{(ax^2 + bx + c)^{k}}\]

\hspace{47pt}Note: A theorem in algebra (which we simply refer to as the partial fraction theorem) \\
\hspace{47pt} \hspace{24pt} guarantees that it is always possible to do this.\\
\indent\\
\indent


Under Step 2, four different cases may occur...\\
\indent\\

\textbf{Case 1: The denominator $\mathbf{Q(x)}$ is a product of distinct linear factors (None repeated).}\\

\[Q(x)=(a_1x+b_1)(a_2x+b_2)\cdots(a_kx+b_k)\]

where no factor is repeated (and no factor is a constant multiple of another). In this case the partial fraction theorem states that there exist constants $A_1,A_2,\ldots,A_k$ such that\\

\[\ds\frac{P(x)}{Q(x)} = \ds\frac{A_1}{a_1x+b_1} + \ds\frac{A_2}{a_2x+b_2} + \cdots + \ds\frac{A_k}{a_kx + b_k}\]

We will see in the following example how these constants can be determined.\\
\indent

\underline{Example 2}: Evaluate $\ds\int \ds\frac{x^2 + 2x-1}{2x^3+3x^2-2x}dx$.\\
\indent

\newpage

\underline{Example 3 (Still under Case 1)} Find $\ds\int \ds\frac{dx}{x^2 - a^2}$, where $a\neq 0$.\\
\indent

\vspace{3.5in}


\textbf{Case 2: $\mathbf{Q(x)}$ is a product of linear factor, some of which are REPEATED.}\\

\[Q(x) = (a_1x+b_1)^r(a_2x+b_2)(a_3x+b_3)\cdots(a_kx+b_k).\]

Then we obtain:
\begin{align*}
\ds\frac{P(x)}{Q(x)} &= \ds\frac{A_1}{a_1x+b_1} + \ds\frac{A_2}{(a_1x+b_1)^2} + \cdots + \ds\frac{A_r}{(a_1x+b_1)^r} +\ds\frac{B}{a_2x+b_2} + \ds\frac{C}{a_3x+b_3} + \cdots + \ds\frac{K}{a_kx + b_k}
\end{align*}

For instance,
\[\ds\frac{x^3 - x + 1}{x^2(x-1)^3} = \hspace{3.5in}\]
\indent
\indent

\underline{Example 4}: Find $\ds\int\ds\frac{x^4 - 2x^2 + 4x+1}{x^3 - x^2 - x + 1}dx$.\\
\indent

\newpage




\textbf{Case 3: $\mathbf{Q(x)}$ contains irreducible quadratic factors, none of which is repeated.}\\

\[Q(x) = (a_1x^2+b_1x+c_1)(a_2x^2+b_2x + c_2)(a_3x+b_3)\cdots(a_kx+b_k)\]
where $b_1^2-4a_1c_1<0$ and $b_2^2-4a_2c_2<0$. Then we obtain:
\[\ds\frac{P(x)}{Q(x)} = \ds\frac{A_1x+B_1}{a_1x^2+b_1x+c_1} + \ds\frac{A_2x+B_2}{a_2x^2 + b_2x+ c_2} + \ds\frac{A_3}{a_3x+b_3} + \cdots + \ds\frac{A_k}{a_kx+b_k}.\]

For instance,

\[f(x) = \ds\frac{P(x)}{Q(x)} = \ds\frac{x}{(x-2)(x^2+1)(x^2+4)} = \hspace{3in}\]

*Note: We can integrate a term of type $\ds\frac{Ax+B}{ax^2+bx+c}$ by the methods of the previous section, 7.3, with completing the square and trig substitution.\\
\indent


\underline{Example 5}: Evaluate $\ds\int \ds\frac{2x^2-x+4}{x^3+4x}dx$.\\
\indent

\newpage

\underline{Example 6}: Evaluate $\ds\int\ds\frac{4x^2-3x+2}{4x^2-4x+3}dx$.\\
\indent

\vspace{3.5in}

\textbf{Case 4: $\mathbf{Q(x)}$ contains a REPEATED irreducible quadratic factor.}\\

\[Q(x)=(ax^2+bx+c)^r\]
Then the PFD of $f(x)$ is:
\[f(x) = \ds\frac{P(x)}{Q(x)} = \ds\frac{A_1x+B_1}{ax^2 + bx + c} + \ds\frac{A_2x+B_2}{(ax^2+bx+c)^2}+\cdots + \ds\frac{A_rx+B_r}{(ax^2+bx+c)^r}\]
\indent

For instance,\\
\indent

$\ds\frac{x^3+x^2+1}{x(x-1)(x^2+x+1)(x^2+1)^3}$\\
\indent\\
\indent

\vspace{10pt}
\indent\\

\vspace{50pt}

\textbf{Note:} Sometimes partial fractions can be avoided when integrating a rational function. For instance, although the integral

\[\ds\int \ds\frac{x^2+1}{x(x^2+3)}dx\]
\indent

could be evaluated by the method of Case (3), it's much easier to observe that:

 \[u=x(x^2+3)=x^3+3x \implies du = (3x^2+3)dx\]
 
 \[\implies \ds\int \ds\frac{x^2+1}{x(x^2+3)}dx = \ds\frac{1}{3} \ds\int \ds\frac{1}{u}du = \ds\frac{1}{3}\ln|x^3+3x| + C\]
 \indent\\
 



\underline{Example 8}: Evaluate $\ds\int \ds\frac{1-x+2x^2-x^3}{x(x^2+1)^2}dx$.\\
\indent

\vspace{3.5in}

\section*{Rationalizing Substitutions}
Some nonrational functions can be changed into rational functions by means of appropriate substitutions. In particular, when an integrand contains an expression of the form $\sqrt[n]{g(x)}$, then the substitution $u=\sqrt[n]{g(x)}$ may be effective.\\
\indent

\underline{Example 9}: Evaluate $\ds\int\ds\frac{\sqrt{x+4}}{x}dx$.\\
\indent











%----------------------------------------------------------------------------------------

\end{document}