%%%%%%%%%%%%%%%%%%%%%%%%%%%%%%%%%%%%%%%%%
% Class Notes Template
% LaTeX Template
% By: Ryan Grove
%%%%%%%%%%%%%%%%%%%%%%%%%%%%%%%%%%%%%%%%%

%----------------------------------------------------------------------------------------
%	PACKAGES AND OTHER DOCUMENT CONFIGURATIONS
%----------------------------------------------------------------------------------------

\documentclass[paper=a4, fontsize=11pt]{scrartcl} % A4 paper and 11pt font size

\usepackage[T1]{fontenc} % Use 8-bit encoding that has 256 glyphs
\usepackage{fourier} % Use the Adobe Utopia font for the document - comment this line to return to the LaTeX default
\usepackage[english]{babel} % English language/hyphenation
\usepackage{amsmath,amsfonts,amsthm} % Math packages

\usepackage{lipsum} % Used for inserting dummy 'Lorem ipsum' text into the template

\usepackage{sectsty} % Allows customizing section commands
\allsectionsfont{\centering \normalfont\scshape} % Make all sections centered, the default font and small caps

\usepackage{fancyhdr} % Custom headers and footers
\pagestyle{fancyplain} % Makes all pages in the document conform to the custom headers and footers
\fancyhead{} % No page header - if you want one, create it in the same way as the footers below
\fancyfoot[L]{} % Empty left footer
\fancyfoot[C]{} % Empty center footer
%\fancyfoot[R]{\thepage} % Page numbering for right footer
\renewcommand{\headrulewidth}{0pt} % Remove header underlines
\renewcommand{\footrulewidth}{0pt} % Remove footer underlines
\setlength{\headheight}{13.6pt} % Customize the height of the header

\numberwithin{equation}{section} % Number equations within sections (i.e. 1.1, 1.2, 2.1, 2.2 instead of 1, 2, 3, 4)
\numberwithin{figure}{section} % Number figures within sections (i.e. 1.1, 1.2, 2.1, 2.2 instead of 1, 2, 3, 4)
\numberwithin{table}{section} % Number tables within sections (i.e. 1.1, 1.2, 2.1, 2.2 instead of 1, 2, 3, 4)

\setlength\parindent{0pt} % Removes all indentation from paragraphs - comment this line for an assignment with lots of text

\usepackage{lastpage}
\usepackage{fancyhdr}
\cfoot{\thepage\ of \pageref{LastPage}}

\def\v{\hbox{$\mathbf v$}}
\def\w{\hbox{$\mathbf w$}}
\def\u{\hbox{$\mathbf u$}}
\def\x{\hbox{$\textbf{x}$}}
\def\z{\hbox{$\mathbf z$}}
\def\a{\hbox{$\mathbf a$}}
\def\b{\hbox{$\mathbf b$}}
\def\L{\hbox{$\mathcal L$}}
\def\C{\hbox{$\mathbb C$}}
\def\B{\hbox{$\mathcal B$}}
\def\R{\hbox{$\mathbb R$}}
\def\X{\hbox{$\underline X$}}
\def\Q{\hbox{$\mathbb Q$}}
\def\R{\hbox{$\mathbb R$}}
\def\N{\hbox{$\mathbb N$}}
\def\C{\hbox{$\mathbb C$}}
\def\0{\hbox{$\mathbf 0$}}
\def\Y{\hbox{$\underline Y$}}
\def\a{\hbox{$\mathbf a$}}
\def\u{\hbox{$\mathbf u$}}
\def\w{\hbox{$\mathbf w$}}
\def\y{\hbox{$\mathbf y$}}
\def\X{\hbox{$\underline X$}}
\def\dd{\hbox{$\partial $}}
\def\B{\hbox{$\mathcal B$}}
\def\F{\hbox{$\mathcal F$}}
\def\L{\hbox{$\mathcal L$}}
\def\M{\hbox{$\mathcal M$}}
\def\D{\hbox{$\mathscr {D}$}}
\def\RR{\hbox{$\mathscr{R}$}}
\def\I{\hbox{$\mathcal I$}}

\usepackage{amssymb}
%\theoremstyle{plain}
\usepackage[margin = .75in]{geometry}
\newtheorem{claim}{Claim}
\newtheorem{theorem}{Theorem}[section]
\newtheorem{lemma}[theorem]{Lemma}
\newtheorem{proposition}[theorem]{Proposition}
\newtheorem{corollary}[theorem]{Corollary}
\newtheorem{problem}[theorem]{Problem}
%\theoremstyle{definition}
\newtheorem{definition}[theorem]{Definition}
%\theoremstyle{remark}
\newtheorem{remark}[theorem]{Remark}
\newtheorem{remarks}[theorem]{Remarks}
\newtheorem{example}[theorem]{Example}
\newcommand{\ds}{\displaystyle}
\newcommand{\ZZ}{\mathbb{Z}}
\newcommand{\QQ}{\mathbb{Q}}
\newcommand{\e}{\varepsilon}
\newcommand{\bbf}{\textbf}
\newcommand{\p}{\parallel}
\usepackage{color}
\newcommand{\field}[1]{\mathbb{#1}}
\usepackage{amsmath}
\usepackage{amsthm}
\usepackage{amssymb}
\usepackage{mathrsfs}
\usepackage{cancel}
\usepackage{upgreek}
\usepackage{graphicx}
\usepackage{multirow}
\usepackage{setspace}
\usepackage{url}
\usepackage{subfigure}
\usepackage{enumerate}
\usepackage{cases}
\usepackage{mathrsfs}
\usepackage{rotating}

%----------------------------------------------------------------------------------------
%	TITLE SECTION
%----------------------------------------------------------------------------------------

\newcommand{\horrule}[1]{\rule{\linewidth}{#1}} % Create horizontal rule command with 1 argument of height

\title{	
\normalfont \normalsize 
\textsc{Ryan Grove, Clemson University, MATH1080 - 9} \\ [25pt] % Your name, university, class
\horrule{0.5pt} \\[0.4cm] % Thin top horizontal rule
\huge Section 11.7: Strategy for Testing Series\\ % The assignment title
\horrule{2pt} \\[0.5cm] % Thick bottom horizontal rule
}

\author{Date:} % The due date

\date{\normalsize March 22, 2016} % A custom date

\begin{document}

\maketitle % Print the title

\begin{flushleft}
\begin{tabular}{l l}
Name: \rule{3.2in}{.01cm}  & {}%Table number: \rule{1in}{.01cm}\\
\end{tabular}
\end{flushleft}

%----------------------------------------------------------------------------------------
%	Lecture
%----------------------------------------------------------------------------------------

\section*{\textbf{Lecture:}}
\begin{enumerate}
\item[1.] If the series is of the form $\ds\sum \ds\frac{1}{n^p}$, it is a $\mathbf{p-}$\textbf{series}, which we know to be convergent if $p>1$ and divergent if $p\leq 1$.\\
\indent

\item[2.] If the series has the form $\ds\sum a r^{n-1}$ or $\ds\sum a r^n$, it is a \textbf{geometric series}, which converges if $|r|<1$ and diverges if $|r|\geq 1$. Some preliminary algebraic manipulation may be required to bring the series into this form.\\
\indent

\item[3.] If the series has a form that is similar to a $p-$series or a geometric series, then one of the \textbf{comparison tests} should be considered. In particular, if $a_n$ is a rational function or an algebraic function of $n$ (involving roots of polynomials), then the series should be compared with a $p-$series. The value of $p$ should be chosen by keeping only the highest powers of $n$ in the numerator and denominator (and simplifying the ratio). The comparison tests apply only to series with positive terms, but if $\ds\sum a_n$ has some negative terms, then we can apply the Comparison Test to $\ds\sum |a_n|$ and test for absolute convergence (which implies convergence).\\
\indent

\item[4.] If you can see at a glance that $\ds\lim\neq 0$, then the \textbf{Test for Divergence} should be used.\\
\indent

\item[5.] If the series is of the form $\ds\sum (-1)^{n-1}b_n$ or $\ds\sum (-1)^n b_n$, then the \textbf{Alternating Series Test} is an obvious possibility.\\
\indent

\item[6.] Series that involve factorials or other products (including a constant raised to the $n^{\text{th}}$ power) are conveniently tested using the \textbf{Ratio Test}.\\
\indent

\item[7.] If $a_n$ is of the form $(b_n)^n$, then the \textbf{Root Test} may be useful.\\
\indent

\item[8.] If $a_n=f(n)$, where $\ds\int_1^\infty f(x) dx$ is easily evaluated, then the \textbf{Integral Test} is effective (assuming the hypotheses of this test are satisfied).\\
\indent

\end{enumerate}


In the following examples we don't work out all the details but simply indicate which test should be used to test convergence/divergence.\\
\indent

\underline{Example 1}: $\ds\sum_{n=1}^\infty \ds\frac{n-1}{2n+1}$\\

\vspace{0.9in}

\underline{Example 2}: $\ds\sum_{n=1}^\infty \ds\frac{\ds\sqrt{n^3+1}}{3n^3 + 4n^2 + 2}$\\

\vspace{1.5in}

\underline{Example 3}: $\ds\sum_{n=1}^\infty ne^{-n^2}$\\

\vspace{0.9in}

\underline{Example 4}: $\ds\sum_{n=1}^\infty (-1)^n \ds\frac{n^3}{n^4+1}$\\

\vspace{0.9in}

\underline{Example 5}: $\ds\sum_{k=1}^\infty \ds\frac{2^k}{k!}$\\

\vspace{0.9in}

\underline{Example 6}: $\ds\sum_{n=1}^\infty \ds\frac{1}{2+3^n}$\\

\vspace{1in}

  
  
  
  
  
  
  


  
  
  
  
  


  

%----------------------------------------------------------------------------------------

\end{document}