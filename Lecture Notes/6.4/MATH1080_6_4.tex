%%%%%%%%%%%%%%%%%%%%%%%%%%%%%%%%%%%%%%%%%
% Class Notes Template
% LaTeX Template
% By: Ryan Grove
%%%%%%%%%%%%%%%%%%%%%%%%%%%%%%%%%%%%%%%%%

%----------------------------------------------------------------------------------------
%	PACKAGES AND OTHER DOCUMENT CONFIGURATIONS
%----------------------------------------------------------------------------------------

\documentclass[paper=a4, fontsize=11pt]{scrartcl} % A4 paper and 11pt font size

\usepackage[T1]{fontenc} % Use 8-bit encoding that has 256 glyphs
\usepackage{fourier} % Use the Adobe Utopia font for the document - comment this line to return to the LaTeX default
\usepackage[english]{babel} % English language/hyphenation
\usepackage{amsmath,amsfonts,amsthm} % Math packages

\usepackage{lipsum} % Used for inserting dummy 'Lorem ipsum' text into the template

\usepackage{sectsty} % Allows customizing section commands
\allsectionsfont{\centering \normalfont\scshape} % Make all sections centered, the default font and small caps

\usepackage{fancyhdr} % Custom headers and footers
\pagestyle{fancyplain} % Makes all pages in the document conform to the custom headers and footers
\fancyhead{} % No page header - if you want one, create it in the same way as the footers below
\fancyfoot[L]{} % Empty left footer
\fancyfoot[C]{} % Empty center footer
%\fancyfoot[R]{\thepage} % Page numbering for right footer
\renewcommand{\headrulewidth}{0pt} % Remove header underlines
\renewcommand{\footrulewidth}{0pt} % Remove footer underlines
\setlength{\headheight}{13.6pt} % Customize the height of the header

\numberwithin{equation}{section} % Number equations within sections (i.e. 1.1, 1.2, 2.1, 2.2 instead of 1, 2, 3, 4)
\numberwithin{figure}{section} % Number figures within sections (i.e. 1.1, 1.2, 2.1, 2.2 instead of 1, 2, 3, 4)
\numberwithin{table}{section} % Number tables within sections (i.e. 1.1, 1.2, 2.1, 2.2 instead of 1, 2, 3, 4)

\setlength\parindent{0pt} % Removes all indentation from paragraphs - comment this line for an assignment with lots of text

\usepackage{lastpage}
\usepackage{fancyhdr}
\cfoot{\thepage\ of \pageref{LastPage}}

\def\v{\hbox{$\mathbf v$}}
\def\w{\hbox{$\mathbf w$}}
\def\u{\hbox{$\mathbf u$}}
\def\x{\hbox{$\textbf{x}$}}
\def\z{\hbox{$\mathbf z$}}
\def\a{\hbox{$\mathbf a$}}
\def\b{\hbox{$\mathbf b$}}
\def\L{\hbox{$\mathcal L$}}
\def\C{\hbox{$\mathbb C$}}
\def\B{\hbox{$\mathcal B$}}
\def\R{\hbox{$\mathbb R$}}
\def\X{\hbox{$\underline X$}}
\def\Q{\hbox{$\mathbb Q$}}
\def\R{\hbox{$\mathbb R$}}
\def\N{\hbox{$\mathbb N$}}
\def\C{\hbox{$\mathbb C$}}
\def\0{\hbox{$\mathbf 0$}}
\def\Y{\hbox{$\underline Y$}}
\def\a{\hbox{$\mathbf a$}}
\def\u{\hbox{$\mathbf u$}}
\def\w{\hbox{$\mathbf w$}}
\def\y{\hbox{$\mathbf y$}}
\def\X{\hbox{$\underline X$}}
\def\dd{\hbox{$\partial $}}
\def\B{\hbox{$\mathcal B$}}
\def\F{\hbox{$\mathcal F$}}
\def\L{\hbox{$\mathcal L$}}
\def\M{\hbox{$\mathcal M$}}
\def\D{\hbox{$\mathscr {D}$}}
\def\RR{\hbox{$\mathscr{R}$}}
\def\I{\hbox{$\mathcal I$}}

\usepackage{amssymb}
%\theoremstyle{plain}
\usepackage[margin = .75in]{geometry}
\newtheorem{claim}{Claim}
\newtheorem{theorem}{Theorem}[section]
\newtheorem{lemma}[theorem]{Lemma}
\newtheorem{proposition}[theorem]{Proposition}
\newtheorem{corollary}[theorem]{Corollary}
\newtheorem{problem}[theorem]{Problem}
%\theoremstyle{definition}
\newtheorem{definition}[theorem]{Definition}
%\theoremstyle{remark}
\newtheorem{remark}[theorem]{Remark}
\newtheorem{remarks}[theorem]{Remarks}
\newtheorem{example}[theorem]{Example}
\newcommand{\ds}{\displaystyle}
\newcommand{\ZZ}{\mathbb{Z}}
\newcommand{\QQ}{\mathbb{Q}}
\newcommand{\e}{\varepsilon}
\newcommand{\bbf}{\textbf}
\newcommand{\p}{\parallel}
\usepackage{color}
\newcommand{\field}[1]{\mathbb{#1}}
\usepackage{amsmath}
\usepackage{amsthm}
\usepackage{amssymb}
\usepackage{mathrsfs}
\usepackage{cancel}
\usepackage{upgreek}
\usepackage{graphicx}
\usepackage{multirow}
\usepackage{setspace}
\usepackage{url}
\usepackage{subfigure}
\usepackage{enumerate}
\usepackage{cases}
\usepackage{mathrsfs}
\usepackage{rotating}

%----------------------------------------------------------------------------------------
%	TITLE SECTION
%----------------------------------------------------------------------------------------

\newcommand{\horrule}[1]{\rule{\linewidth}{#1}} % Create horizontal rule command with 1 argument of height

\title{	
\normalfont \normalsize 
\textsc{Ryan Grove, Clemson University, MATH1080 - 9} \\ [25pt] % Your name, university, class
\horrule{0.5pt} \\[0.4cm] % Thin top horizontal rule
\huge Section 6.4: Work  \\ % The assignment title
\horrule{2pt} \\[0.5cm] % Thick bottom horizontal rule
}

\author{Date:} % The due date

\date{\normalsize January 19, 2016} % A custom date

\begin{document}

\maketitle % Print the title

\begin{flushleft}
\begin{tabular}{l l}
Name: \rule{3.2in}{.01cm}  & {}%Table number: \rule{1in}{.01cm}\\
\end{tabular}
\end{flushleft}

%----------------------------------------------------------------------------------------
%	Lecture
%----------------------------------------------------------------------------------------

\section*{\textbf{Lecture:}}

In Physics, \underline{\hspace{1in}} depends on \underline{\hspace{1in}}. Intuitively, you can think of a force as describing a push or pull on an object. In general, if an object moves along a straight line with position function $s(t)$, then the \textbf{force} $F$ on the object (in the same direction) is given by Newton's Second Law of Motion as the product of its mass $m$ and its acceleration:\\
\indent


\[\underline{\hspace{2in}}\]
\indent\\

In the SI metric system, the mass is measured in kilograms (kg), the displacement in meters (m), the time in seconds (s), and the force in newtons (N$=$ kg$\cdot$m$/\text{s}^2$). Thus a force of 1 N acting on a mass of 1 kg produces an acceleration of 1 m$/\text{s}^2$. In the US Customary system the fundamental unit is chosen to be the unit of force, which is the \underline{\hspace{1.25in}}.\\
\indent

In the case of constant acceleration, the force $F$ is also constant and the work done is defined to be the product of the force $F$ and the distance $d$ that the object moves:\\
\indent

\[\underline{\hspace{1.5in}} \quad \quad \underline{\hspace{2.5in}}\]
\indent\\

If $F$ is measured in newtons and $d$ in meters, then the unit for $W$ is a newton-meter, which is called a joule (J). If $F$ is measured in pounds and $d$ in feet, then the unit for $S$ is a foot-pound (ft-lb), which is about 1.36 J.\\
\indent\\
\newpage
\underline{Example 1}:
\begin{enumerate}
\item[(a)] How much work is done in lifting a 1.2-kg book off the floor to put it on a desk that is 0.7 m high? Use the fact that the acceleration due to gravity is $g=9.8\text{m}/\text{s}^2$.\\
\indent\\

\vspace{1.25in}

\item[(b)] How much work is done in lifting a 20-lb weight 6 ft off the ground?


\vspace{1.25in}

\end{enumerate}

*Notice that in (b), unlike part (a), we did not have to multiply by $g$ because we were given the \textit{weight} (which is a force) and not the mass of the object.\\
\indent\\

What happens if the force is not constant, but instead variable? Let's suppose the object moves along the $x-$axis in the positive direction, from $x=a$ to $x=b$, and at each point $x$ between $a$ and $b$ a force $f(x)$ acts on the object, where $f$ is a continuous function. If we divide the interval $[a,b]$ into $n$ subintervals with endpoints $x_0, x_1, \ldots, x_n$ and equal width $\Delta x$, and choose a sample point $x_i^*$ in the $i^{\text{th}}$ subinterval $[x_{i-1},x_i]$. Then the force at that point is $f(x_i^*)$. Assuming $n$ is relatively large, then $\Delta x$ is small, and since $f$ is continuous, the values of $f$ don't change very much over the interval $[x_{i-1},x_i]$. In other words, $f$ is almost constant on the interval and so the work $W_i$ that is done in moving the particle from $x_{i-1}$ to $x_i$ is approximately given by:\\
\indent\\

\[\underline{\hspace{2in}}\]
\indent\\

Thus we can approximate the total work by:\\
\indent\\

\[\underline{\hspace{2in}}\]

Now, as $n\to \infty$, this approximation gets better and better, so we define:\\
\indent

\fbox{
  \parbox{\textwidth}{
  \vspace{5pt}
\textbf{Definition:} The \textbf{work done in moving the object from} $\mathbf{a}$ \textbf{to} $\mathbf{b}$ is:\\
\indent\\

\[\underline{\hspace{2in}} = \underline{\hspace{1.5in}}\]
\indent

}}
\indent\\
\indent\\

\underline{Example 2}: When a particle is located a distance $x$ feet from the origin, a force of $x^2 + 2x$ pounds acts on it. How much work is done in moving it from $x=1$ to $x=3$?\\
\indent

\vspace{1.25in}

For the next example we need the following law from physics:\\


\begin{quote}
\underline{\hspace{1.25in}} \underline{\hspace{0.6in}} states that the force required to maintain a spring stretched $x$ units beyond its natural length is proportional to $x$:\\

\[\underline{\hspace{1.5in}}\]
\indent\\

where $k$ is a positive constant (called the \textbf{spring constant}). Hooke's Law holds provided that $x$ is not too large.\\
\end{quote}
\indent\\

\underline{Example 3}: A force of 40 N is required to hold a spring that has been stretched from its natural length of 10 cm to a length of 15 cm. How much work is done in stretching the spring from 15 cm to 18 cm?\\
\indent

\textbf{Solution:} According to Hooke's Law, the force required to hold the spring stretched $x$ meters beyond its natural length is \\

\[\underline{\hspace{1.5in}}\]
\indent\\

When the spring is stretched from 10 cm to 15 cm, the amount stretched is \underline{\hspace{1.6in}}.\\
\indent\\
\indent\\
\indent\\

Thus $f(x) = \underline{\hspace{0.75in}}$ and the work done in stretching the spring from 15 cm to 18 cm is\\
\indent\\

\vspace{1.5in}

\underline{Example 4}: A 200-lb cable is 100 ft long and hangs vertically from the top of a tall building. How much work is required to lift the cable to the top of the building?\\
\indent

\textbf{Solution:} Here we don't have a formula for the force function, but we can use an interval argument similar to what led to the definition above.\\
\indent\\

\vspace{3.5in}


%----------------------------------------------------------------------------------------

\end{document}