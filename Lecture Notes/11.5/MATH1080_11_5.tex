%%%%%%%%%%%%%%%%%%%%%%%%%%%%%%%%%%%%%%%%%
% Class Notes Template
% LaTeX Template
% By: Ryan Grove
%%%%%%%%%%%%%%%%%%%%%%%%%%%%%%%%%%%%%%%%%

%----------------------------------------------------------------------------------------
%	PACKAGES AND OTHER DOCUMENT CONFIGURATIONS
%----------------------------------------------------------------------------------------

\documentclass[paper=a4, fontsize=11pt]{scrartcl} % A4 paper and 11pt font size

\usepackage[T1]{fontenc} % Use 8-bit encoding that has 256 glyphs
\usepackage{fourier} % Use the Adobe Utopia font for the document - comment this line to return to the LaTeX default
\usepackage[english]{babel} % English language/hyphenation
\usepackage{amsmath,amsfonts,amsthm} % Math packages

\usepackage{lipsum} % Used for inserting dummy 'Lorem ipsum' text into the template

\usepackage{sectsty} % Allows customizing section commands
\allsectionsfont{\centering \normalfont\scshape} % Make all sections centered, the default font and small caps

\usepackage{fancyhdr} % Custom headers and footers
\pagestyle{fancyplain} % Makes all pages in the document conform to the custom headers and footers
\fancyhead{} % No page header - if you want one, create it in the same way as the footers below
\fancyfoot[L]{} % Empty left footer
\fancyfoot[C]{} % Empty center footer
%\fancyfoot[R]{\thepage} % Page numbering for right footer
\renewcommand{\headrulewidth}{0pt} % Remove header underlines
\renewcommand{\footrulewidth}{0pt} % Remove footer underlines
\setlength{\headheight}{13.6pt} % Customize the height of the header

\numberwithin{equation}{section} % Number equations within sections (i.e. 1.1, 1.2, 2.1, 2.2 instead of 1, 2, 3, 4)
\numberwithin{figure}{section} % Number figures within sections (i.e. 1.1, 1.2, 2.1, 2.2 instead of 1, 2, 3, 4)
\numberwithin{table}{section} % Number tables within sections (i.e. 1.1, 1.2, 2.1, 2.2 instead of 1, 2, 3, 4)

\setlength\parindent{0pt} % Removes all indentation from paragraphs - comment this line for an assignment with lots of text

\usepackage{lastpage}
\usepackage{fancyhdr}
\cfoot{\thepage\ of \pageref{LastPage}}

\def\v{\hbox{$\mathbf v$}}
\def\w{\hbox{$\mathbf w$}}
\def\u{\hbox{$\mathbf u$}}
\def\x{\hbox{$\textbf{x}$}}
\def\z{\hbox{$\mathbf z$}}
\def\a{\hbox{$\mathbf a$}}
\def\b{\hbox{$\mathbf b$}}
\def\L{\hbox{$\mathcal L$}}
\def\C{\hbox{$\mathbb C$}}
\def\B{\hbox{$\mathcal B$}}
\def\R{\hbox{$\mathbb R$}}
\def\X{\hbox{$\underline X$}}
\def\Q{\hbox{$\mathbb Q$}}
\def\R{\hbox{$\mathbb R$}}
\def\N{\hbox{$\mathbb N$}}
\def\C{\hbox{$\mathbb C$}}
\def\0{\hbox{$\mathbf 0$}}
\def\Y{\hbox{$\underline Y$}}
\def\a{\hbox{$\mathbf a$}}
\def\u{\hbox{$\mathbf u$}}
\def\w{\hbox{$\mathbf w$}}
\def\y{\hbox{$\mathbf y$}}
\def\X{\hbox{$\underline X$}}
\def\dd{\hbox{$\partial $}}
\def\B{\hbox{$\mathcal B$}}
\def\F{\hbox{$\mathcal F$}}
\def\L{\hbox{$\mathcal L$}}
\def\M{\hbox{$\mathcal M$}}
\def\D{\hbox{$\mathscr {D}$}}
\def\RR{\hbox{$\mathscr{R}$}}
\def\I{\hbox{$\mathcal I$}}

\usepackage{amssymb}
%\theoremstyle{plain}
\usepackage[margin = .75in]{geometry}
\newtheorem{claim}{Claim}
\newtheorem{theorem}{Theorem}[section]
\newtheorem{lemma}[theorem]{Lemma}
\newtheorem{proposition}[theorem]{Proposition}
\newtheorem{corollary}[theorem]{Corollary}
\newtheorem{problem}[theorem]{Problem}
%\theoremstyle{definition}
\newtheorem{definition}[theorem]{Definition}
%\theoremstyle{remark}
\newtheorem{remark}[theorem]{Remark}
\newtheorem{remarks}[theorem]{Remarks}
\newtheorem{example}[theorem]{Example}
\newcommand{\ds}{\displaystyle}
\newcommand{\ZZ}{\mathbb{Z}}
\newcommand{\QQ}{\mathbb{Q}}
\newcommand{\e}{\varepsilon}
\newcommand{\bbf}{\textbf}
\newcommand{\p}{\parallel}
\usepackage{color}
\newcommand{\field}[1]{\mathbb{#1}}
\usepackage{amsmath}
\usepackage{amsthm}
\usepackage{amssymb}
\usepackage{mathrsfs}
\usepackage{cancel}
\usepackage{upgreek}
\usepackage{graphicx}
\usepackage{multirow}
\usepackage{setspace}
\usepackage{url}
\usepackage{subfigure}
\usepackage{enumerate}
\usepackage{cases}
\usepackage{mathrsfs}
\usepackage{rotating}

%----------------------------------------------------------------------------------------
%	TITLE SECTION
%----------------------------------------------------------------------------------------

\newcommand{\horrule}[1]{\rule{\linewidth}{#1}} % Create horizontal rule command with 1 argument of height

\title{	
\normalfont \normalsize 
\textsc{Ryan Grove, Clemson University, MATH1080 - 9} \\ [25pt] % Your name, university, class
\horrule{0.5pt} \\[0.4cm] % Thin top horizontal rule
\huge Section 11.5: Alternating Series\\ % The assignment title
\horrule{2pt} \\[0.5cm] % Thick bottom horizontal rule
}

\author{Date:} % The due date

\date{\normalsize March 8, 2016} % A custom date

\begin{document}

\maketitle % Print the title

\begin{flushleft}
\begin{tabular}{l l}
Name: \rule{3.2in}{.01cm}  & {}%Table number: \rule{1in}{.01cm}\\
\end{tabular}
\end{flushleft}

%----------------------------------------------------------------------------------------
%	Lecture
%----------------------------------------------------------------------------------------

\section*{\textbf{Lecture:}}
The convergence tests that we have looked at so far apply only to series with positive terms. In this section and the next we learn how to deal with series whose terms are not necessarily positive. Of particular importance are \textit{alternating series}, whose terms alternate in sign.\\
\indent

An \underline{\hspace{1.25in}} \underline{\hspace{1in}} is a series whose terms are alternating positive and negative. Here are two examples:
\begin{align*}
1 - \ds\frac{1}{2} + \ds\frac{1}{3} - \ds\frac{1}{4} + \ds\frac{1}{5} - \ds\frac{1}{6} + \cdots = \ds\sum_{n=1}^\infty (-1)^{n-1}\ds\frac{1}{n}\\
-\ds\frac{1}{2} + \ds\frac{2}{3} - \ds\frac{3}{4} + \ds\frac{4}{5} - \ds\frac{5}{6} + \ds\frac{6}{7} - \cdots = \ds\sum_{n=1}^\infty (-1)^n\ds\frac{n}{n+1}\\
\end{align*}

We see from these examples that the $n^{\text{th}}$ term of an alternating series is of the form

\[a_n = (-1)^{n-1}b_n \quad \text{ or } \quad a_n = (-1)^n b_n\]\\
\indent

where $b_n$ is a positive number. (In fact, $b_n=|a_n|$.)\\
\indent

The following test says that if the terms of an alternating series decrease toward 0 in absolute value, then the series converges.\\
\indent

\fbox{
  \parbox{\textwidth}{
  \vspace{5pt} \textbf{Alternating Series Test}:\\
  \indent
  
  If the alternating series\\
  
  \[\ds\sum_{n=1}^\infty(-1)^{n-1}b_n = b_1 - b_2 + b_3 - b_4 + b_5 - b_6 + \cdots \quad b_n >0\]
  \indent
  
  satisfies
  \begin{align*}
  (i) \text{ } & b_{n+1}\leq b_n \quad \text{ for all } n\\
  \text{ }\\
  (ii) \text{ } & \ds\lim_{n\to\infty}b_n = 0\\
  \end{align*}
  
  then the series is convergent.\\
  
  }}
  \indent\\
  \indent
  
  (Proof omitted. See page 728 of textbook.)
  \indent\\
  \indent
  
  \underline{Example 1}: The alternating harmonic series:
  
  \[1-\ds\frac{1}{2} + \ds\frac{1}{3} - \ds\frac{1}{4} + \cdots = \ds\sum_{n=1}^\infty \ds\frac{(-1)^{n-1}}{n}\]
  
  satisfies
  \vspace{-20pt}
  \begin{align*}
  (i) \text{ } &b_{n+1}<b_n \quad \text{ because } \quad \ds\frac{1}{n+1}<\ds\frac{1}{n} \hspace{2in}\\
  (ii) \text{ } &\ds\lim_{n\to\infty}b_n = \ds\lim_{n\to\infty}\ds\frac{1}{n} =0
  \end{align*}
  
  so the series is convergent by the Alternating Series Test.\\
  \indent\\
  \indent
  
  \underline{Example 2}: The series $\ds\sum_{n=1}^\infty \ds\frac{(-1)^n 3n}{4n-1}$ is alternating, but
  
  \vspace{0.8in}
  
  so condition $(ii)$ is not satisfied. Instead, we look at the limit of the $n^{\text{th}}$ term of the series:\\
  \indent
  
  \vspace{1.25in}
  
  \indent
  \newpage
  
  \underline{Example 3}: Test the series $\ds\sum_{n=1}^\infty (-1)^{n+1}\ds\frac{n^2}{n^3+1}$ for convergence or divergence.\\
  \indent
  
  \newpage
  
  \section*{Estimating Sums}
  
  Recall that a partial sum, $S_n$, of any convergent series can be used as an approximation to the total sum, $S$, but this is not of much use unless we can estimate the accuracy of the approximation. The error involved in using $S\approx S_n$ is the remainder $R_n = $ \underline{\hspace{1in}}. The next theorem says that for series that satisfy the conditions of the Alternating Series Test, the size of the error is smaller than \underline{\hspace{0.5in}}, which is the absolute value of the first neglected term.\\
  \indent
  
  \fbox{
  \parbox{\textwidth}{
  \vspace{5pt} \textbf{Alternating Series Estimation Theorem}:\\
  \indent
  
  If $S=\ds\sum (-1)^{n-1}b_n$ is the sum of an alternating series that satisfies
  
  \[(i) b_{n+1}\leq b_n \quad \text{ and } \quad (ii) \ds\lim_{n\to\infty}b_n = 0\]
  
  then,
  \[|R_n| = |S-S_n| \leq b_{n+1}\]
  \indent
  
  }}
  \indent\\
  \indent
  
  \underline{Example 4}: Find the sum of the series $\ds\sum_{n=0}^\infty \ds\frac{(-1)^n}{n!}$ correct to three decimal places.\\
  \indent
  
  \vspace{4.25in}
  
  \textbf{NOTE}: The rule that the error (in using $S_n$ to approximate $S$) is smaller than the first neglected term is, in general, valid only for alternating series that satisfy the conditions of the Alternating Series Estimation Theorem. The rule does NOT apply to other types of series.\\
  \indent
  
  
  
  
  
  
  


  
  
  
  
  


  

%----------------------------------------------------------------------------------------

\end{document}