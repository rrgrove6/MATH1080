%%%%%%%%%%%%%%%%%%%%%%%%%%%%%%%%%%%%%%%%%
% In-class Quiz Template
% LaTeX Template
% By: Ryan Grove
%%%%%%%%%%%%%%%%%%%%%%%%%%%%%%%%%%%%%%%%%

%----------------------------------------------------------------------------------------
%	PACKAGES AND OTHER DOCUMENT CONFIGURATIONS
%----------------------------------------------------------------------------------------

\documentclass[paper=a4, fontsize=11pt]{scrartcl} % A4 paper and 11pt font size

\usepackage[T1]{fontenc} % Use 8-bit encoding that has 256 glyphs
\usepackage{fourier} % Use the Adobe Utopia font for the document - comment this line to return to the LaTeX default
\usepackage[english]{babel} % English language/hyphenation
\usepackage{amsmath,amsfonts,amsthm} % Math packages

\usepackage{lipsum} % Used for inserting dummy 'Lorem ipsum' text into the template

\usepackage{sectsty} % Allows customizing section commands
\allsectionsfont{\centering \normalfont\scshape} % Make all sections centered, the default font and small caps

\usepackage{fancyhdr} % Custom headers and footers
\pagestyle{fancyplain} % Makes all pages in the document conform to the custom headers and footers
\fancyhead{} % No page header - if you want one, create it in the same way as the footers below
\fancyfoot[L]{} % Empty left footer
\fancyfoot[C]{} % Empty center footer
%\fancyfoot[R]{\thepage} % Page numbering for right footer
\renewcommand{\headrulewidth}{0pt} % Remove header underlines
\renewcommand{\footrulewidth}{0pt} % Remove footer underlines
\setlength{\headheight}{13.6pt} % Customize the height of the header

\numberwithin{equation}{section} % Number equations within sections (i.e. 1.1, 1.2, 2.1, 2.2 instead of 1, 2, 3, 4)
\numberwithin{figure}{section} % Number figures within sections (i.e. 1.1, 1.2, 2.1, 2.2 instead of 1, 2, 3, 4)
\numberwithin{table}{section} % Number tables within sections (i.e. 1.1, 1.2, 2.1, 2.2 instead of 1, 2, 3, 4)

\setlength\parindent{0pt} % Removes all indentation from paragraphs - comment this line for an assignment with lots of text

\usepackage{lastpage}
\usepackage{fancyhdr}
\cfoot{\thepage\ of \pageref{LastPage}}

\usepackage{graphicx}

%----------------------------------------------------------------------------------------
%	TITLE SECTION
%----------------------------------------------------------------------------------------

\newcommand{\horrule}[1]{\rule{\linewidth}{#1}} % Create horizontal rule command with 1 argument of height

\title{	
\normalfont \normalsize 
\textsc{Ryan Grove, Clemson University, MATH1080 - 9} \\ [25pt] % Your name, university, class
\horrule{0.5pt} \\[0.4cm] % Thin top horizontal rule
\huge In-class Quiz \#14 \\ % The assignment title
\horrule{2pt} \\[0.5cm] % Thick bottom horizontal rule
}

\author{Date:} % The due date

\date{\normalsize April 5, 2016} % A custom date

\begin{document}

\maketitle % Print the title

\begin{flushleft}
\begin{tabular}{l l}
Name: \rule{3.2in}{.01cm}  & {}%Table number: \rule{1in}{.01cm}\\
\end{tabular}
\end{flushleft}

%----------------------------------------------------------------------------------------
%	Directions
%----------------------------------------------------------------------------------------

\section*{\textbf{Directions:}}

No calculator or notes may be used.  Read each question very carefully.  In order to receive full credit, you must:
\begin{enumerate}
\item Show legible and logical (relevant) justification which supports your final answer.
\item Use complete and correct mathematical notation.
\item Include proper units, if necessary.
\item Give exact numerical values whenever possible.
\item Follow the directions given for the problem.
\end{enumerate}
\vspace{.1in}

\newpage

\begin{enumerate}
\item (2 points) Use the test of your choice to determine whether 
$\sum\limits_{n=1}^\infty \sin \left( \frac{1}{n} \right)$ is convergent or divergent.\\
\noindent\textbf{Solution:}\\
Are the terms positive? Yes.\\
For $n \ge 1$, then $0 \le \frac{1}{n} \le 1$.  Since $\sin x > 0$ for $0 < x < \pi$, then $\sin \frac{1}{n} \ge 0$. Use the Limit Comparison Test and compare the series to the divergent Harmonic Series, $\sum\limits_{n=1}^\infty \frac{1}{n}$. \\

Using the trigonometric limit $\lim\limits_{x \rightarrow \infty} \frac{\sin \frac{1}{x}}{\frac{1}{x}}=1$ and by applying L'Hopital's Rule,  $\lim\limits_{x \rightarrow \infty} \frac{\sin \frac{1}{x}}{\frac{1}{x}}>0$. Therefore, by the Limit Comparison Test, the series $\boxed{ \text{diverges}}$.


\newpage

\item (6 points) Consider the series $\sum\limits_{n=1}^\infty \frac{1}{n^2}$.
\begin{enumerate}
\item Determine if the integral test can be applied to this series.\\
\noindent\textbf{Solution:}\\
Let $f(x)=\frac{1}{x^2}$. Observe that $f(x)$ is positive and continuous for $x \ge 1$.  To determine if $f(x)$ is decreasing, compute its derivative.  $f'(x)=\frac{-2}{x^3}<0$ for $x>0$.  Then $f'(x)<0$ when $x \ge 1$.  $\boxed{ \text{The three conditions for the integral test are satisfied.}}$
\vspace{.25in}
\item If the integral test can be applied, use it to determine if the series converges or diverges.\\
\noindent\textbf{Solution:}\\
\begin{eqnarray*}
\int\limits_1^\infty \frac{1}{x^2} \text{ } dx & = & \lim\limits_{b \rightarrow \infty} \int\limits_1^b \frac{1}{x^2} \text{ } dx \\
& = & \lim\limits_{b \rightarrow \infty}  -\frac{1}{x} |_1^b \\
& = & \lim\limits_{b \rightarrow \infty} 
\left[ -\frac{1}{b} + 1 \right] \\
& = & 1
\end{eqnarray*}

The improper integral converges. By the Integral Test, the series $\sum\limits_{n=1}^\infty \frac{1}{n^2}$ $\boxed{ \text{converges}}$.
\vspace{.25in}
\newpage
\item Approximate the sum of the series using the first three terms of the series. Use the Remainder Estimate for the integral test to bound (lower and upper) the error for this approximation.
\\
\noindent\textbf{Solution:}\\
\end{enumerate}
$S \approx S_3 = 1 + \frac{1}{4} + \frac{1}{9} = \frac{49}{36}$.  Since $R_3=S-S_3$.  Then
\begin{equation*}
\int\limits_4^\infty f(x) \text{ } dx \le R_3 \le \int\limits_3^\infty f(x) \text{ } dx
\end{equation*}
\begin{equation*}
\int\limits_4^\infty f(x) \text{ } dx = \lim_{b \rightarrow \infty} \int\limits_n^b x^{-2} \text{ } dx = \lim_{b \rightarrow \infty} \left( -x^{-1} \right)|^b_n = \lim_{b \rightarrow \infty} \left( -\frac{1}{b}+\frac{1}{n} \right) = \frac{1}{n}
\end{equation*}
Therefore, $\boxed{\frac{1}{4} \le R_3 \le \frac{1}{3}}$.

\vspace{1.5in}

\item (2 points) Use the test of your choice to determine whether 
$\sum\limits_{n=1}^\infty (-1)^n \sin \left( \frac{\pi}{n} \right)$ is convergent or divergent.\\
\noindent\textbf{Solution:}\\
$b_n=\sin \left( \frac{\pi}{n} \right)>0$ for $n \ge 2$ and $\sin \left( \frac{\pi}{n} \right) \ge \sin \left( \frac{\pi}{n+1} \right)$, and $\lim\limits_{n \rightarrow \infty} \sin \left( \frac{\pi}{n} \right) = \sin 0 = 0$, so the series converges by the Alternating Series Test.

\newpage

\item (2 points) Use the test of your choice to determine whether 
$\sum\limits_{n=1}^\infty \frac{n!}{3^n n^2}$ is convergent or divergent.\\
\noindent\textbf{Solution:}\\
The terms of the series are positive.  Because of the factorial use the Ratio Test.
\begin{equation*}
\frac{a_{n+1}}{a_n}=\frac{\frac{(n+1)!}{3^{n+1}(n+1)^2}}{\frac{n!}{3^n n^2}}
\end{equation*}

\begin{eqnarray*}
\lim\limits_{n \rightarrow\infty} \frac{\frac{(n+1)!}{3^{n+1}(n+1)^2}}{\frac{n!}{3^n n^2}}&=&\lim\limits_{n \rightarrow\infty} \frac{(n+1)!3^n n^2}{n! 3^{n+1}(n+1)^2}\\
&=&\lim\limits_{n \rightarrow\infty} \frac{(n+1) n^2}{ 3(n+1)^2}\\
&=&\lim\limits_{n \rightarrow\infty} \frac{n^3+n^2}{ 3n^2+6n+3} \\
&=&\lim\limits_{n \rightarrow\infty} \frac{n+1}{ 3+\frac{6}{n}+\frac{3}{n^2}}  \\
&=& \infty
\end{eqnarray*}

$\boxed{ \text{By the Ratio Test the series diverges.}}$

\newpage

\item (2 points) Use the test of your choice to determine whether 
$\sum\limits_{n=3}^\infty \cos (n \pi) \tan (\frac{\pi}{n})$ is convergent or divergent. \textbf{Hint:} Write out a few terms.\\
\noindent\textbf{Solution:}\\
\begin{equation*}
\sum\limits_{n=3}^\infty \cos (n \pi) \tan (\frac{\pi}{n}) = -\tan (\frac{\pi}{3}) + \tan (\frac{\pi}{4}) -\tan (\frac{\pi}{5}) + ... = \sum\limits_{n=3}^\infty (-1)^n \tan (\frac{\pi}{n})
\end{equation*}
Use the Alternating Series Test:  Note that $b_n = |(-1)^n \tan (\frac{\pi}{n})| =\tan (\frac{\pi}{n}) $
Two conditions must hold: \\
$1: \lim\limits_{n \rightarrow\infty} \tan \left( \frac{\pi}{n} \right) = 0$ \\
For $n \ge 3$, then $0<\frac{\pi}{n} \le \frac{\pi}{3}$.  Since $\tan x > 0$ for $0 < x \le \frac{\pi}{3}$, then $\tan \frac{\pi}{n} \ge 0$.\\

$2: \tan \left( \frac{\pi}{n+1}  \right) \le \tan \left( \frac{\pi}{n}  \right) $ for all $n \ge 3$. \\
$n+1 > n$ for $n \ge 3$\\
$\frac{\pi}{n+1} =\frac{\pi}{n}$ \\
$\tan \left( \frac{\pi}{n+1} \right) = \tan \left( \frac{\pi}{n} \right)$ since $\tan x$ is an increasing function\\

By the Alternating Series Test, the series $\sum\limits_{n=3}^\infty \cos (n \pi) \tan (\frac{\pi}{n})$ $\boxed{\text{converges}}$.

\newpage

\item (6 points) Consider the series $\sum\limits_{n=1}^\infty \frac{(-1)^{n+1}}{\sqrt{n}}$.
\begin{enumerate}
\item Does the series converge absolutely?\\
\noindent\textbf{Solution:}\\
 $\sum\limits_{n=1}^\infty |\frac{(-1)^{n-1}}{\sqrt{n}}|=\sum\limits_{n=1}^\infty \frac{1}{\sqrt{n}}$, which is a divergent p-series.  Therefore the series $\sum\limits_{n=1}^\infty \frac{(-1)^{n+1}}{\sqrt{n}}$ does \boxed{\text{not converge absolutely.}}
 \vspace{.5in}
\item Does the series converge conditionally?\\
\noindent\textbf{Solution:}\\
Use the Alternating Series Test to determine if the series $\sum\limits_{n=1}^\infty \frac{(-1)^{n+1}}{\sqrt{n}}$ converges. 

Note that $b_n=|\frac{(-1)^{n+1}}{\sqrt{n}}|=\frac{1}{\sqrt{n}}.$  Two conditions must hold:

1. $\lim\limits_{n \rightarrow \infty} \frac{1}{\sqrt{n}}=0.$\\

2. $\frac{1}{\sqrt{n+1}} \le \frac{1}{\sqrt{n}}$ for all $n\ge 1$.\\
Let $f(x)=\frac{1}{\sqrt{x}}$.  Then $f'(x)=\frac{1}{2}x^{-3/2}$.  Therefore $f(x)$ is decreasing for $x > 0$ and $0<b_{n+1}\le b_n$ for $n \ge 1$, that is the $b_n$ are non-increasing.

By the Alternating Series Test, the series  $\sum\limits_{n=1}^\infty \frac{(-1)^{n+1}}{\sqrt{n}}$ converges.  Since it does not converge absolutely, it \boxed{\text{converges conditionally.}}

 \vspace{.5in}
\item What are the fewest number of terms necessary to approximate the sum of the series with an error bound of no more than $0.01$?\\
\noindent\textbf{Solution:}\\
Note that since our series is alternating, we can just look at the first neglected term, i.e. look for the first $a_n$ that satisfies $|a_n|<0.01$.

$|R_n|=|S-S_n|\le b_{n+1}$\\
$b_{n+1}=\frac{1}{\sqrt{n+1}}\le \frac{1}{100}$\\
$100\le \sqrt{n+1}$\\
$\implies n \ge (100)^2-1$

The fewest number of terms necessary to approximate the sum of the series with an error bound of no more than $0.01$ is $\boxed{9999 \text{ terms.}}$

\end{enumerate}

\end{enumerate}

%----------------------------------------------------------------------------------------

\end{document}
